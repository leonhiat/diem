\SubSection{Function Condition Injection}

During specification injection, move specifications are reduced to basic
assume/assert statements added to the Move code.  Those statements represent
instructions to the solver backend about what propositions can be assumed and
which need to be asserted (verified) at a given program point.  In this section,
we cover how \emph{function specification conditions} are injected.


\SubSubSection{Modular Verification}
\label{sec:ModularVerification}

Modular verification applies to all types of injections, and its principles are
therefore described first. When the Move prover is run, it takes as input a set
of Move modules which is closed under the transitive dependency relation (module
imports). However, only a subset of those modules are \emph{verification target}
(typically just one module). It is assumed that the tool environment ensures
that modules in the dependency relation which are not target of verification
have already successfully verified. This is possible since Move has an acyclic
import relation.

From the set of target modules, the set of \emph{target functions} is
derived. This set might be enriched by additional functions which need
verification because of global invariants, as discussed in
Sec.~\ref{sec:GlobalInvariants}. The resulting set of target functions will then
be verified one-by-one, assuming that any called functions have successfully
verified. If a called function is among the target functions, it might in fact
not verify; however, in this case a verification error will be reported at the
called function, and the verification result at the caller side can be ignored.

\SubSubSection{Pre- and Post conditions}

The injection of basic function specifications is illustrated in
Fig.~\ref{fig:RequiresEnsuresAbortsIf}.  An extension of the Move source
language is used to specify abort behavior. With~%
|fun f() { .. } onabort { conditions }| a Move function is defined where
|conditions| are assume or assert statements that are evaluated at every program
point the function aborts (either implicitly or with an |abort| statement). This
construct simplifies the presentation and corresponds to a per-function abort
block on bytecode level which is target of branching.

\begin{Figure}
  \caption{Requires, Ensures, and AbortsIf Injection}
  \label{fig:RequiresEnsuresAbortsIf}
  \centering
\begin{MoveBox}
  fun f(x: u64, y: u64): u64 { x + y }
  spec f {
    requires x < y;
    aborts_if x + y > MAX_U64;
    ensures result == x + y;
  }
  fun g(x: u64): u64 { f(x, x + 1) }
  spec g {
    ensures result > x;
  }
  @\transform@
  fun f(x: u64, y: u64): u64 {
    spec assume x < y;
    let result = x + y;
    spec assert result == x + y;     // ensures of of
    spec assert                      // negated abort_if of f
      !(x + y > MAX_U64); @\label{line:aborts_holds_not}@
    result
  } onabort {
    spec assert                      // abort_if of f
      x + y > MAX_U64; @\label{line:aborts_holds}@
  }
  fun g(x: u64): u64 {
    spec assert x < x + 1;           // requires of f
@$\textrm{\it if inlined}$\label{line:inline}@
    let result = inline f(x, x + 1);
@$\textrm{\it elif opaque}$\label{line:opaque}@
    if (x + x + 1 > MAX_U64) abort;  // aborts_if of f
    spec assume result == x + x + 1; // ensures of f
@$\textrm{\it endif}$@
    spec assert result > x;          // ensures of g
    result
  }
\end{MoveBox}
\end{Figure}

An aborts condition is translated into two different asserts: one where the
function aborts and the condition must hold (line~\ref{line:aborts_holds}), and
one where it returns and the condition must \emph{not} hold
(line~\ref{line:aborts_holds_not}). If there are multiple |aborts_if|, they are
or-ed. If there is no aborts condition, no asserts are generated. This means
that once a user specifies aborts conditions, they must completely cover the
abort behavior of the code. (The prover also provides an option to relax this
behavior, where aborts conditions can be partial and are only enforced on
function return.)

For a function call site we distinguish two variants: the call is \emph{inlined}
(line~\ref{line:inline}) or it is \emph{opaque} (line~\ref{line:opaque}). In
both cases, it is assumed that the called function is verified (see Modular
Verification, Sec.~\ref{sec:ModularVerification}). For inlined calls, the
function definition, with all injected assumptions and assertions turned into
assumptions (as those are considered proven) is substituted. For opaque
functions the specification conditions are inserted as
assumptions. Methodologically, opaque functions need precise specifications
relative to a particular objective, where as in the case of inlined functions
the code is still the source of truth and specifications can be partial or
omitted. However, inlining does not scale arbitrarily, and can be only used for
small function systems.

Notice we have not discussed the way how to deal with relating pre and post
states yet, which requires taking snapshots of state (e.g.~%
|ensures x == old(x) + 1|); the example in
Fig.~\ref{fig:RequiresEnsuresAbortsIf} does not need it. Snapshots of state
will be discussed for global update invariants in Sec.~\ref{sec:GlobalInvariants}.

\SubSubSection{Modifies}

\begin{Figure}
  \caption{Modifies Injection}
  \label{fig:Modifies}
  \centering
\begin{MoveBox}
  fun f(addr: address) { move_to<T>(addr, T{}) }
  spec f {
    pragma opaque;
    ensures exists<T>(addr);
    modifies global<T>(addr);
  }
  fun g() { f(0x1) }
  spec g {
    modifies global<T>(0x1); modifies global<T>(0x2);
  }
  @\transform@
  fun f(addr: address) {
    let can_modify_T = {addr};         // modifies of f
    spec assert addr in can_modify;    // permission check move_to @%
                                            \label{line:modifies_permission}@
    move_to<T>(addr, T{});
  }
  fun g() {
    let can_modify_T = {0x1, 0x2};     // modifies of g
    spec assert {0x1} <= can_modify_T; // permission check call f @%
                                            \label{line:modifies_call_permission}@
    spec havoc global<T>(0x1);         // havoc memory modified by f @%
                                            \label{line:modifies_havoc}@
    spec assume exists<T>(0x1);        // ensures of f
  }
\end{MoveBox}
\end{Figure}


The |modifies| condition specifies that a function only changes specific memory.
It comes in the form |modifies global<T>(addr)|, and its injection is
illustrated in Fig.~\ref{fig:Modifies}.

A type check is used to ensure that if a function has one or more~%
|modifies| conditions all called functions which are \emph{opaque} have a matching
modifies declaration. This is important so we can relate the callees
memory modifications to that what is allowed at caller side.

At verification time, when an operation is performed which modifies memory, an
assertion is emitted that modification is allowed
(e.g. line~\ref{line:modifies_permission}). The permitted addresses derived from
the modifies clause are stored in a set |can_modify_T| generated by the
transformation. Instructions which modify memory are either primitives (like
|move_to| in the example) or function calls. If the function call is inlined,
modifies injection proceeds (conceptually) with the inlined body. For opaque
function calls, the static analysis has ensured that the target has a modifies
clause.  This clause is used to derive the modified memory, which must be a
subset of the modified memory of the caller
(line~\ref{line:modifies_call_permission}).

For opaque calls, we also need to \emph{havoc} the memory they modify
(line~\ref{line:modifies_havoc}), by which is meant assigning an unconstrained
value to it. If present, |ensures| from the called function, injected as
subsequent assumptions, are further constraining the modified memory.


\SubSubSection{Data Invariants}

\begin{Figure}
  \caption{Data Invariant Injection}
  \label{fig:DataInvariants}
  \centering
\begin{MoveBox}
  struct S { a: u64, b: u64 }
  spec S { invariant a < b }
  fun f(s: S): S { let r = &mut s; r.a = r.a + 1; r.b = r.b + 1; s }
  @\transform@
  fun f(s: S): S {
    spec assume s.a < s.b;      // assume invariant for parameter
    let r = Mvp::local(s, F_s); // begin mutation of s
    r = Mvp::set(r, Mvp::get(r)[a = Mvp::get(r).a + 1]);
    r = Mvp::set(r, Mvp::get(r)[b = Mvp::get(r).b + 1]);
    spec assert                 // end mutation: invariant enforced
      Mvp::get(r).a < Mvp::get(r).b;
    s = Mvp::get(r);            // write back to s
    s
  }
\end{MoveBox}
\end{Figure}

A data invariant specifies a constraint over a struct value. The value is
guaranteed to satisfy this constraint at any time. Thus, when a value is
constructed, the data invariant needs to be verified, and when it is consumed,
it can be assumed to hold.

In Move's reference semantics, construction of struct values is often done via a
sequence of mutations via mutable references. It is desirable that \emph{during}
such mutations, assertion of the data invariant is suspended. This allows to
state invariants which reference multiple fields, where the fields are updated
step-by-step.  Move's borrow semantics and concept of mutations provides a
natural way how to defer invariant evaluation: at the point a mutable reference
is released, mutation ends, and the data invariant can be enforced.  In other
specification formalisms, we would need a special language construct for
invariant suspension. Fig.~\ref{fig:DataInvariants} gives an example, and shows
how data invariants are reduced to assert/assume statements.

\Paragraph{Implementation}

The implementation hooks into the reference elimination
(Sec.~\ref{sec:RefElim}). As part of this the lifetime of references is
computed. Whenever a reference is released and the mutated value is written
back, we also enforce the data invariant. In addition, the data invariant is
enforced when a struct value is directly constructed.



\SubSection{Global Invariant Injection}
\label{sec:GlobalInvariants}

Global invariants appear on Move module level and constraint the content of the
memory. While the basic injection of global invariants is relative simple, they
cause significant complexity with features like modular verification, suspension,
and generics. We first discuss the basic model, then extend it step wise.

\SubSubSection{Basic Translation}

\begin{Figure}
  \caption{Basic Global Invariant Injection}
  \label{fig:GlobalInvariants}
  \centering
\begin{MoveBox}
  fun f(a: address) {
    let r = borrow_global_mut<S>(a);
    r.value = r.value + 1
  }
  invariant [I1] forall a: address: global<S>(a).value > 0;
  invariant [I2] update
      forall a: address: global<S>(a).value > old(global<S>(a).value);
  @\transform@
  fun f(a: address) {
    spec assume I1;
    Mvp::snapshot_state(I2_BEFORE);
    r = <increment mutation>;
    spec assert I1;
    spec assert I2[old = I2_BEFORE];
  }
\end{MoveBox}
\end{Figure}

Fig.~\ref{fig:GlobalInvariants} contains an example for the supported invariant
types and their injection into code. The first invariant, |I1|, is a regular
state invariant. It is assumed on function entry, and asserted after the state
update. The second, |I2|, is a state update invariant, which relates pre and
post states. For this a state snapshot is stored under some label |I2_BEFORE|,
which is then used in an assertion.

Global invariant injection is optimized by knowledge of the prover, obtained
by static analysis, about (transitively) accessed memory. For opaque functions
(including also builtin functions) this information is obtained via the modifies
clause. For other functions it is determined from the code.  Assuming that the
prover has precise knowledge (up to symbolic address representation) of memory
usage, it can determine which invariants to inject. Let |f| be a target function:

\begin{itemize}
\item Inject |assume I| at entry to |f| \emph{if} |read*(f)| has overlap with
  |read*(I)|.
\item At every point in |f| where a memory location |M| is updated inject
  |assert I| after the update \emph{if} |M in read*(I)|. Also, if |I| is an
  update invariant, before the update inject a memory snapshot save.
\end{itemize}

Notice that we do not inject any invariants in functions which are not
verification target. However, the set of target functions may need to be
extended because of invariants, as described later.


\SubSubSection{Genericity}

\begin{Figure}
  \caption{Genericity}
  \label{fig:Genericity}
  \centering
\begin{MoveBox}
  invariant [I1] global<S<u64>>(0).value > 1;
  invariant<T> [I2] global<S<T>>(0).value > 0;
  fun f(a: address) { borrow_global_mut<S<u8>>(0).value = 2 }
  fun g<R>(a: address) { borrow_global_mut<S<R>>(0).value = 3 }
  @\transform@
  fun f(a: address) {
    spec assume I2[T = u8];
    <<mutate>>
    spec assert I2[T = u8];
  }
  fun g<R>(a: address) {
    spec assume I1;
    spec assume I2[T = R];
    <<mutate>>
    spec assert I1;
    spec assert I2[T = R];
  }
\end{MoveBox}
\end{Figure}

In the case of generic invariants and functions, we must use \emph{type
  unification} to determine which invariants are injected. Consider the example
in Fig.~\ref{fig:Genericity}. Invariant |I1| holds for a specific type
instantiation |S<u64>|, whereas |I2| is generic over all type instantiations for
|S<T>|.

The non-generic function |f| which works on the instantiation |S<u8>| will have
to inject the \emph{specialized} instance |I2[T = u8]|. The invariant |I1|,
however, does not apply for this function, because there is no overlap with
|S<u64>|.  In contrast, in the generic function |g| we have to inject both
invariants. Because this function works on arbitrary instances, it is also
relevant for the specific case of |S<u64>|.

In the general case, we are looking at a unification problem of the following
kind. Given the accessed memory of a function |f<R>| and an invariant |I<T>|, we
compute the pairwise unification of memory types. Those types are parameterized
over |R| resp. |T|, and successful unification will result in a substitution
for both. On successful unification, we include the invariant with |T| specialized
according to the substitution.

Notice that there are implications related to monomorphization coming from the
injection of global invariants; those are discussed in Sec.~\ref{sec:Mono}.


\SubSubSection{Modularity}

\begin{Figure}
  \caption{Modular Verification and Invariants}
  \label{fig:ModularVerificationInv}
  \centering
\begin{MoveBox}
  module Store {
    struct T has key { x: u64 }
    public fun read(): u64 { borrow_global<S>(0).x }
    public fun write(x: u64) { borrow_global_mut_<S>(0).x = x }
  }
  module Actor {
    use Store;
    invariant global<S>(0).x > 0;
    public fun update(x: u64) {
      if (x == 0) then abort 1;
      Store::set(x);
    }
  }
\end{MoveBox}
\end{Figure}

In Sec.~\ref{sec:ModularVerification}, the general mechanism of modular
verification was described, deriving the set of verified \emph{target
  functions} from the set of \emph{target modules}, provided by the user on the
command line. Global invariants add additional functions by possibly requiring
re-verification of non-target functions which can influence the invariant.

Consider the example in Fig.~\ref{fig:ModularVerificationInv}. The module
|Store| provides an API for some storage location which is shared between a set
of modules. The module |Actor|, one of those modules, establishes an
invariant on the content of the store. When |Actor| is verified, one must
also verify the function |Store::write|, because this invariant is verification
target.  (In this example, verification cannot succeed, because the function
|Store::write| is not restricting the values for the parameter |x|; we see in
the next section how to fix this.)

In general, the set of additional functions to verify is computed as
follows. Let |I| be a target invariant which appears in some target module, and
|f| some function in the dependency relation. If |modify(f)| has an overlap
with |read*(I)| then |f| needs to be added to the target functions. Notice it is
not |modify*(f)|; only direct modifications make a function to a verification
target (with exceptions as discussed in the next section).


\SubSubSection{Suspending Invariants}

\begin{Figure}
  \caption{Suspension of Invariants}
  \label{fig:SuspensionInv}
  \centering
\begin{MoveBox}
  module Store {
    friend Actor;
    ...
    public(friend) fun write(x: u64) { borrow_global_mut_<S>(0).x = x }
    spec write { pragma suspend_invariants; }
  }
  module Actor {
    ...
    invariant [suspendable] global<S>(0).x > 0;
  }
\end{MoveBox}
\end{Figure}

The example in Fig.~\ref{fig:ModularVerificationInv} is not quite right from a
design viewpoint, since a global store accessible to everybody is constrained by
a specific module. Consequently, it cannot be successfully
verified. Fig.~\ref{fig:SuspensionInv} modifies the example to fix this. First,
Move's |friend| mechanism is used to restrict visibility of |Store::write| to
the |Actor| module. Note one could add other modules to the friends list as
needed.  Second, the |Store::write| function is declared to \emph{suspend
  invariant evaluation to callers}. Only private and friend functions can have
such a declaration, ensuring the all call sites are known and the suspended
invariants are actually verified in all call contexts.  An invariant needs to be
explicitly marked via |[suspendable]| do be eligible for suspension.

When an invariant |I| is suspended for a function |F|, the injection scheme changes
as follows:

\begin{itemize}
\item At the definition side of |F|, |I| is neither assumed nor asserted.
\item At every call side of |F| (whether opaque or inlined), the invariant
  is asserted right after the call. It will also be assumed at the entry point
  of the caller.
\item Instead of |F| becoming a target function because it modifies the memory
  read in |I| (see above paragraph about modular verification), all callers will
  become target functions.
\item If the caller is itself suspended, the process is instead continued with
  the parent callers.
\end{itemize}

Once a function is suspended, automatically all functions it calls which modify
memory effected by the suspended invariants are suspended as well. This is
because when those functions are called, the relevant invariants cannot be
assumed to hold, and therefore it is likely not fruitful to try to verify
something related to them.

For update invariants it should be noted that suspension may change their
meaning, depending on the form of the predicate. Without suspension, an update
invariant is implemented by snapshotting the memory before the update and then
asserting a predicate after the update which refers to the previous state and
the current one. For suspended update invariants, the snapshot is taken
\emph{before} the suspended function is called, and the assertion injected
\emph{after} it returns, which might be earlier resp.~later states. An example
of an update invariant which works well for suspension is e.g. a requirement for
a monotonically increasing value, as in~%
|invariant [suspendable] old(value()) <= value()|.

Methodologically, the suspension mechanism should be used with care, because it
may complicate the verification problem by propagating verification errors to
more complex application contexts. The Move prover supports a further pragma to
suspend invariant verification which draws a clear boundary to function systems
with suspension. With |pragma suspend_invariants_in_body| a function can be
marked to suspend invariants only in its body but ensure they hold at caller
side. This is conceptually syntactic sugar for introducing a helper function:

\begin{Move}
  public fun f(P) { S } spec f { pragma suspend_invariants_in_body; }
  @transform@
  public fun f(P) { f'(P) }
  fun f'(P) { S } spec f' { pragma suspend_invariants; }
\end{Move}



\SubSubSection{Invariant Consistency}

\TODO{wrwg}{Describe solution to the below problem via induction}

Notice that invariant injection can lead to inconsistencies. Consider the following
code fragment:

\begin{Move}
  invariant [I] forall a: address: global<S>(a).value > 0;
  @\transform@
  spec assume global<S>(0).value == 0;  // context, e.g. from a requires
  spec assume I;                        // injected
\end{Move}

\noindent We currently do not check whether an invariant is satisfiable
before we assume it, but rather rely on a generic consistency checker for
specifications.


%%% Local Variables:
%%% mode: latex
%%% TeX-master: "main"
%%% End:
