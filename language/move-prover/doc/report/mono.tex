\SubSection{Monomorphization}
\label{sec:Mono}

Monomorphization is the process of removing all generics from a Move programs.
It greatly improves the performance of the backend solvers.

\SubSubSection{Generic Functions}

\begin{figure}[t!]
\caption{Basic Monomorphization}
\label{fig:Mono}
\centering
\begin{MoveBox}
  fun f<T>(x: T) { g<S<T>>(S(x)) }
  @\transform@
  struct given_T{}
  fun f_T(x: given_T) { g_S_T(S_T(x)) }
\end{MoveBox}
\end{figure}

To verify a generic function, monomorphization skolemizes the type parameter
into a given type. It then, for all functions which are inlined, inserts their
code specializing it for the given type instantiation, including specialization
of all used types. Fig.~\ref{fig:Mono} sketches this approach.

The underlying conjecture is that if we verify |f<given_T>|, we have also
verified for all possible instantiations. However, this statement is
only correct for code which does not depend on runtime type information.

\SubSubSection{Type Dependent Code}

In Move, types are (almost) not able to influence runtime semantics. There is
one exception: if memory is indexed by a generic, as in |S<T>|. One can
essentially implement a generic type check in move:

\begin{Move}
  fun init() { move_to<S<u64>>(s, S{} }
  fun is_u64<T>(): bool { exists<S<u64>> }
\end{Move}

\noindent The important property enabling monomorphization is that we can
identify such dependencies by looking at the memory accessed by code (and
injected specifications). Assume that a function |f<T>| accesses memory |S<T>|.
If the same function also access any instantiation of this memory (say |S<u64>|),
we need to deal with the case that |S<T>| and |S<u64>| overlap in the effects.
Specifically, if we just monmorphize into |f_T| which uses |S_T| and |S_u64|, we
miss any conditions dependent on the case that |T = u64|. Consider the
following code fragment:

\begin{Move}
  fun f<T>() { move_to<S<T>>(s, ..); move_to<S<u64>>(s, ..) }
\end{Move}

\noindent This function aborts in the case that |T = u64|, but not necessary if |T != u64|.

The solution to this problem is that we verify not only |f_T| but also
|f_u64|. In general, the set of monomorphized instances of a function |f<T>|
which need to be verified is determined by finding all instantiations of |T|
that some pair of memory accesses in the function can overlap.

\TODO{}{formalization and proof?}

Notice that even though it is not common in regular Move code to work with both
memory |S<T>| and, say, |S<u64>| in one function, there is a common scenario
where such code is implicitly created by injection of global
invariants. Consider the example in Fig.~\ref{fig:Genericity}. The invariant
|I1| which works on |S<u64>| is injected into the function |g<R>| which works on
|S<R>|. When monomorphizing |g|, we need to verify an instance |g_u64| in order
to ensure that |I1| holds.



%%% Local Variables:
%%% mode: latex
%%% TeX-master: "main"
%%% End:
