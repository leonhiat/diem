
\usepackage{xcolor}
\usepackage{listings}
\usepackage{inconsolata}
\usepackage{amsmath,amssymb,array}
\usepackage{mathpartir}
\usepackage{graphicx}

\newcommand{\Section}[1]{\section{#1}}
\newcommand{\SubSection}[1]{\subsection{#1}}
\newcommand{\SubSubSection}[1]{\subsubsection{#1}}
\newcommand{\Paragraph}[1]{\paragraph{#1}}

\newcommand{\TODO}[2]{%
  \noindent\textcolor{gray}{%
      {\scriptsize\textbf{TODO(#1)}: #2}
  }
}

\newcommand{\MVP}[0]{\textsf{MVP}}


% Source Transformation

\newcommand{\transform}[0]{\Large{$\leadsto$}}


%%%% Code

\lstdefinestyle{MoveStyle}{
  basicstyle=\ttfamily,
  keywordstyle=\bf,
  escapechar=@, % use to embed LaTeX into code
}


\lstdefinelanguage{Move}{
  morekeywords={
    abort,
    aborts_if,
    acquires,
    address,
    as,
    assert,
    assume,
    borrow_global,
    borrow_global_mut,
    break,
    const,
    continue,
    copy,
    copyable,
    define,
    drop,
    else,
    ensures,
    exists,
    false,
    forall,
    fun,
    global,
    has,
    havoc,
    if,
    include,
    invariant,
    key,
    let,
    loop,
    modifies,
    module,
    move,
    move_from,
    move_to,
    mut,
    native,
    num,
    old,
    onabort,
    pragma,
    public,
    requires,
    resource,
    return,
    schema,
    script,
    signer,
    spec,
    store,
    struct,
    true,
    u64,
    update,
    use,
    with,
    where,
    while},
  sensitive=true,
  morecomment=[l]{//},
  morecomment=[s]{/*}{*/},
}

% This allows us to use |<move code>| for inline code.
\lstMakeShortInline[language=Move,style=MoveStyle]|


% This defines a new environment for Move code.
\lstnewenvironment{Move}{
  \lstset{
    language=Move,
    style=MoveStyle,
    basicstyle=\footnotesize\ttfamily,
  }
}{
}

% This defines a new environment for Move code in a box, with line numbers.
% TODO: this is intended for figures, but there are some weird problems
% with how lstlisting does things which are not fully understood. So figure
% needs to be done manually around this.
\lstnewenvironment{MoveBox}{
  \lstset{
    language=Move,
    style=MoveStyle,
    basicstyle=\footnotesize\ttfamily,
    numbers=left,
    numberstyle=\scriptsize\color{gray},
    frame=single,
  }
}{
}

% This defines a new environment for diagnostics as produced by the prover.
\lstnewenvironment{MoveDiag}{
  \lstset{
    style=MoveStyle,
    basicstyle=\scriptsize\ttfamily,
  }
}{
}


%%%%%%%%%%%%%%%%%%%%%%%%%%%%%%%%%%%%%%%%%%%%%%%%%%%%%%%%%%%%%%%%%%%%
% Operational Semantics
% (older stuff, needed for formal.tex)

\newcommand{\smem}{\mathcal{M}}
\newcommand{\saddr}{\mathcal{A}}
\newcommand{\sval}{\mathcal{V}}
\newcommand{\sresource}{\mathcal{R}}
\newcommand{\slocal}{\mathcal{L}}
\newcommand{\sconst}{\mathcal{C}}
\newcommand{\sfield}{\mathcal{F}}
\newcommand{\sop}{\mathcal{O}}
\newcommand{\sexp}{\mathcal{E}}
\newcommand{\sstatelabel}{\mathcal{S}}

\newcommand{\svmerror}{\kappa}

\newcommand{\sestate}{\Theta}
\newcommand{\serel}[3]{#1 \vdash #2 \Rightarrow #3}
\newcommand{\sereltwo}[3]{\begin{array}{r}#1 \vdash #2 \\ \Rightarrow #3\end{array}}
\newcommand{\seabort}[3]{{#1}_{\mathrm{abort}(#3)~\mathrm{if}~#2}}
\newcommand{\seval}{\mathrm{eval}}
\newcommand{\sref}[2]{\mathrm{ref}(#1, #2)}
\newcommand{\srootg}[2]{#1_{#2}}


% Types
\newcommand{\stype}{\mathcal{T}}
\newcommand{\stctor}[1]{\mathrm{#1}}
\newcommand{\stbool}{\stctor{bool}}
\newcommand{\stnum}{\stctor{num}}
\newcommand{\stresource}{\stctor{resource}}
\newcommand{\staddress}{\stctor{address}}
\newcommand{\stref}{\sbctr{ref}}

\newcommand{\stmut}{\mathcal{M}}
\newcommand{\smut}{\stctor{mut}}
\newcommand{\simut}{\stctor{imu}}

% Definitions
\newcommand{\sdef}{\mathcal{D}}
\newcommand{\sdfun}[3]{\sbinstr{fun}(#1)#2 \{ #3 \}}

% Basic Bytecodes
\newcommand{\sbinstr}[1]{\mathtt{#1}}
\newcommand{\sbytecode}{\mathcal{B}}
\newcommand{\sbload}{\sbinstr{load}}
\newcommand{\sbassign}{\sbinstr{assign}}
\newcommand{\sbpack}{\sbinstr{pack}}
\newcommand{\sbunpack}{\sbinstr{unpack}}
\newcommand{\sbop}{\sbinstr{op}}
\newcommand{\sbmoveto}{\sbinstr{move\_to}}
\newcommand{\sbmovefrom}{\sbinstr{move\_from}}
\newcommand{\sbexists}{\sbinstr{exists}}
\newcommand{\sbborrowl}{\sbinstr{borrow\_local}}
\newcommand{\sbborrowg}{\sbinstr{borrow\_global}}
\newcommand{\sbborrowf}{\sbinstr{borrow\_field}}
\newcommand{\sbwriteref}{\sbinstr{write\_ref}}
\newcommand{\sbreadref}{\sbinstr{read\_ref}}
\newcommand{\sbcall}{\sbinstr{call}}
\newcommand{\sbabort}{\sbinstr{abort}}
\newcommand{\sbreturn}{\sbinstr{return}}
\newcommand{\sbrelease}{\sbinstr{release}}
\newcommand{\sbif}{\sbinstr{if}}
\newcommand{\sbwhile}{\sbinstr{while}}


% Imutable References Elimination
\newcommand{\sbreadl}{\sbinstr{read\_local}}
\newcommand{\sbreadg}{\sbinstr{read\_global}}
\newcommand{\sbselectf}{\sbinstr{select\_field}}
\newcommand{\sitrafo}[2]{#1 \leadsto #2}
\newcommand{\sitrafotwo}[2]{\begin{array}{r}#1 \\ \leadsto #2\end{array}}

% Specification constructs
\newcommand{\sbassume}{\sbinstr{assume}}
\newcommand{\sbassert}{\sbinstr{assert}}
\newcommand{\sbensures}{\sbinstr{ensures}}




% General Math
\newcommand{\sseq}[1]{\overline{#1}}
\newcommand{\sgen}[1]{\langle#1\rangle}
\newcommand{\scross}[0]{\times}
\newcommand{\swbot}[1]{#1_\bot}
\newcommand{\sinto}[0]{\rightarrow}
\newcommand{\swith}[0]{\mathit{with~}}
\newcommand{\supdate}[3]{#1[#2 \triangleleft #3]}
\newcommand{\scat}{\cdot}
\newcommand{\sif}{~\mathrm{if}~}
\newcommand{\selse}{~\mathrm{else}~}
\newcommand{\sget}[2]{#1_#2}
\newcommand{\sset}[3]{#1[#2 \triangleleft #3]}


\newenvironment{leftgather}{\begin{math}\begin{array}{l}}{\end{array}\end{math}}

%%% Local Variables:
%%% mode: latex
%%% TeX-master: "main"
%%% End:
